\chapter{Benutzeranleitung}
\label{Benutzeranleitung}

\section{Vorraussetzungen}

\subsection{Binary}
Um das vorkompilierte Binary auszuführen ist ein 64bit Linux System notwendig.

\subsection{Kompilation}
Vorraussetzungen zur Kompilation des Programmes ist eine go in Version >= 1.11.
Für die Ausführung der Unittests ebenfalls.
Das Programm unterstütz go modules.

Die Kompilation erfolget mit dem
\begin{lstlisting}
    build.sh
\end{lstlisting}  Skript oder dem Befehl
\begin{lstlisting}
    go build .
\end{lstlisting}

\section{Ausführung}

Die Ausführung geschieht durch das aufrufen der Binary, durch das
\begin{lstlisting}
    run.sh
\end{lstlisting}  Skript oder dem Befehl
\begin{lstlisting}
    go run .
\end{lstlisting}

Alle Varianten erfordern als Parameter den Pfad zu einer Datei oder einen Ordner mit ".txt" Dateien.

\section{Tests}

Die Tests können mit dem
\begin{lstlisting}
    run:_tests.sh
\end{lstlisting}  Skript oder dem Befehl
\begin{lstlisting}
    go test --cover ./...
\end{lstlisting}

ausgeführt werden.
