\chapter{Verfahrensbeschreibung}
\label{chap:Verfahrensbeschreibung}

\section{Übersicht}

Zuerst werden die Staaten und Nachbarschaftsbeziehungen aus der Eingabedatei eingelesen.
Danach wird mithilfe eines iterativen Algorithmus mit einer festen Anzahl von Iterationen eine Lösung gesucht.
Diese wird dann in eine für gnuplot aufbereitete Form in eine Ausgabedatei geschrieben.

\section{Einlesen}

Das Einlesen geschieht mit einem speziell auf das Eingabeformat abgestimmten Parser der die Eingabedatei nur genau einmal durchläuft.
Der Parser ist so geschrieben das er die Eingabe nicht nur aus Dateien lesen kann sondern ist über das io.Reader interface aus der golang Standardbibliothek erweiterbar.

\section{Algorithmus}

Der Algorithmus besteht aus einer festen Anzahl von Iterationen. In jeder Iterationen werden die Mittelpunkte der Staaten verschoben.
Diese Verschiebungen geschehen in Abhängigkeit von den aktuellen Überschneidungen sowie den Abständen von Nachbarstaaten.
Dafür werden zuerst für jeden Staat Abstoßungskräfte durch Überschneidungen und Anziehungskräfte durch Nachbarn berechnet,
dann diese pro Staat addiert und schlussendlich alle Staaten proportional zur berechneten Kraft verschoben.
Diese Verschiebung ist gedämpft um die echten Lagebeziehungen möglichst beizubehalten und um Schwingungseffekte zu verringern.

\subsection{Berechnung der Kräfte}

Zur Berechnung der Kräfte wird über jeden Staat(a) iteriert und der Abstand zu jedem anderen Staat(b) berechnet,
hierbei ist der kürzeste Abstand von Kreisaußenkante zu Kreisaußenkante gemeint und nicht der Abstand der Mittelpunkte.
Falls die Kreise sich überschneiden ist der Abstand negativ.
Abhängig davon ob die Länder benachbart sind wird dann eine Kraft auf den Staat berechnet.
Die Stärke der Kraft(f) abhängig vom Abstand(d) berechnet sich bei Nachbarstaaten wie folgt:
\begin{center}
    $$ f(d) = d $$
\end{center}
Dies sorgt dafür das Nachbarn sich berühreren.
Bei nicht Nachbarschaftsbeziehungen:
\begin{center}
    $
    f(d) = \left\{
        \begin{array}{ll}
            1,2 \cdot d & \textrm{falls }d < 0 \\
            0 & \textrm{sonst} \\
        \end{array} \right.
    $
\end{center}
Dies sorgt dafür das nicht benachbarte Länder sich nicht überschneiden.
Die Richtung der Kraft ist hierbei immer vom Mittelpunkt des Staates a in Richtung des Mittelpunktes Staat b.
Bei einer negativen Stärke wirkt die Kraft in die umgekehrte Richtung.
Da die Kraft von b auf a auch gleichzeitig eine gleichstarke Kraft von a auf b bewirkt,
werden alle Kräfte halbiert damit die Abstände durch sie nicht doppelt kompensiert werden.

\subsection{Anwendung der Kräfte}

Nachdem alle Kräfte berechnet wurden wird über alle Staaten iteriert und auf jeden die berechnete Kraft(f) angewendet.
Dies geschieht durch eine Verschiebung(v):

\begin{center}
    $ v = c \cdot f , c \epsilon (0,1) $
\end{center}

Wobei c ein Dämpfungsfaktor zwischen 0 und 1 ist.
Die Dämpfung sorgt dafür das die Lagebeziehungen nicht sicht nicht zu stark ändern und soll Schwingungseffekte verhindern.

\section{Komplexität}

Dadurch das für jeden Staat der Abstand zu jedem anderen Staat berechnet werden muss, liegt die Komplexität des Algorithmus bei $O(n)^2$ wobei n die Anzahl der Staaten ist.

\section{Ausgabe}

Die Ausgabe ist getrennt von dem Algorithmus und bereitet die Ergebnisse für gnuplot auf.
