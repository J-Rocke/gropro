\chapter{Aufgabenanalyse}
\label{chap:Aufgabenanalyse}

\section{Analyse}
In der Aufgabe geht es darum schemenhafte Karten auf Basis von vorgegebenen Kennwerten zu erstellen.
Dabei werden Staaten als Kreise dargestellt und die Fläche der Kreise ist proportional zum Kennwert.
Die Schwierigkeit liegt hierbei eine möglichst überschneidungsfreie Darstellung zu finden
und gleichzeitig die urspünglichen Lage- und Nachbarschaftsbeziehungen möglichst beizubehalten.

\section{Eingabeformat}

Das Programm erwartet den Pfad zu einer Datei oder Verzeichniss enthält als Kommandozeilenparameter.
Falls der übergebene Pfad ein Verzeichniss ist werden alle Dateien in diesem Verzeichniss die mit .txt enden als Eingaben verwendet und hintereinander abgearbeitet.
Falls nur eine Datei übergeben wird muss sie mit .txt enden.

Die Eingabedateien werden nach folgenden Formatsvorgaben eingelesen. Bei Abweichungen von der Vorgabe ist das Verhalten undefiniert oder eine Fehlermeldung wird ausgegeben.

\subsection{Formatsforgaben}

\subsubsection{Kommentare}
Alle Zeilen die mit einem oder mehreren \# Zeichen beginnen werden als Kommentare ignoriert.

\subsubsection{Titel}
Die erste Zeile wird als Titel des Kennwertes gepeichert.

\subsubsection{Staaten}

Die darauf folgenden Zeilen beschreiben die Staaten die in der Karte dargestellt werden.
Sie müssen aus einer ID, einem Kennwert, einem Längengrad und einem Breitengrad bestehen. Die einzelnen Werte müssen durch eine beliebige Anzahl von Whitespace getrennt sein.
Die ID muss ein string sein, der Kennwert eine ganze positive Zahl und Längengrad und Breitengrad eine Fließkommazahl.
Es werden so lange Staaten eingelesen bis eine Zeile ein : Symbol enthält.

\subsubsection{Nachbarschaften}

Alle weiteren Zeilen müssen mit der ID eines Staates gefolgt von einem Doppelpunkt(:), gefolgt von einer Liste von IDs bestehen.
Diese Zeilen bilden die Nachbarschaftsbeziehungen der Staaten ab.
Nachbarschaften sind symetrisch das heißt wenn z.B. NL ein Nachbar von D ist auch D ein Nachbar von NL.
Es reicht jedoch wenn in der Eingabe die Nachbarschaft nur in eine richtung enthalten ist. Die Rückrichtung gilt dann automatisch auch.

\subsection{Beispiel}

\lstinputlisting{../tests/bierkonsum.txt}

\section{Ausgabeformat}
Die Ausgabe erfolgt in eine Datei oder meherere Dateien. Die Ausgabe ist speziell auf das Programm gnuplot ab Version 5 zugeschnitten.
Diese Dateien haben den gleichen Namen wie die Eingabedateien doch eine .gnu Endung anstatt der .txt Endung.

\subsection{Beschreibung}
Das Format der Ausgabedateien ist wie folgt:

\lstinputlisting{./ausgabeformat.gnu}

Dabei werden die <Tags> wie folgt ersetzt.



\begin{center}
    \begin{tabular}{ c c }
        Tag & Inhalt  \\
        \hline
        <xmin> & Kleinster X-Wert im Darstellungsbereich  \\
        <xmax> & Größter X-Wert im Darstellungsbereich  \\
        <ymin> & Kleinster Y-Wert im Darstellungsbereich  \\
        <ymax> & Größter Y-Wert im Darstellungsbereich  \\
        <titel> & Der eingelesene Titel  \\
        <nr> & Anzahl der Iterationen  \\
        <Liste aus ....> & Liste aus Staaten, je Zeile ein Staat. \\
        <xpos> & X-Koordinate des Kreismittelpunktes  \\
        <ypos> & Y-Koordinate des Kreismittelpunktes \\
        <ID> & ID des Staates  \\
        <numericID> & Forlaufende numerische ID (dient zur Färbung)  \\
    \end{tabular}
    \end{center}

\subsection{Beispiel}

\lstinputlisting{../tests/bierkonsum.gnu}

\section{Speziallfälle}

Folgende Spezialfälle wurden identifizeirt und sind getestet.

\begin{itemize}
    \item Nur ein Staat
    \item TODO
\end{itemize}

\section{Fehlerfälle}

Im Falle eines Fehler durch falsche Eingaben wird eine Fehlermeldung ausgegeben und die Lösungsfindung für die entsprechende Eingabe abgebrochen.

\section{Vereinfachungen}

Gegenüber der Realität unserer Erde wurden hierbei eine Reihe von vereinfachten Ahnnahmen getroffen.
\begin{enumerate}
    \item Die Karte ist unendlich groß.
    \item Die Karte ist flach und nicht gewölbt.
    \item Der Abstand zwischen zwei Längengraden ist immer gleich groß.
    \item Der Abstand benachbarter Längengrade ist gleich dem Abstand benachbarter Breitengrade
    \item Daraus folgend sind Abstände nicht auf einer Kugel berechnet.
\end{enumerate}
